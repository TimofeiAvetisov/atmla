\documentclass{article}
\usepackage[utf8]{inputenc}
\usepackage[T2A]{fontenc} % Кодировка для кириллицы
\usepackage[russian,english]{babel} % Поддержка русского языка
\usepackage{soul} % Для перечеркнутого текста
\usepackage{hyperref} % Для гиперссылок
\usepackage{amsmath} % Для математики
\usepackage{geometry} % Для контроля за полями документа
\geometry{a4paper, margin=1in} % Задает поля

\title{Документация по проекту с матрицами}
\author{Timofei Avetisov}
\date{}

\begin{document}

\maketitle

\section{Вступление}
Данный проект был написан за несколько дней, так что не советую читать \st{говно} код. Но с его помощью можно делать разные операции с матрицами, которые обычно лень выполнять вручную.

\section{Как этим пользоваться}
Для начала необходимо скачать всё с \href{https://github.com/TimofeiAvetisov/atmla.git}{GitHub}. Откройте проект и приготовьтесь. Работать нужно только с файлом \texttt{main.cpp}. В начале обязательно напишите:
\begin{verbatim}
#include "Header/Matrix/Matrix.h"
\end{verbatim}
Это автоматически подключит \texttt{<bits/stdc++.h>}. Если нужно что-то ещё — смело подключайте. Далее пишите код.

Немного расскажу прo \texttt{atmla::Fraction}. У класса есть два поля - numerator, denominarator. Конструкторы есть следующие:
\begin{enumerate}
    \item \texttt{atmla::Fraction f(int n, int m)} - создаст дробь $\frac{n}{m}$
    \item \texttt{atmla::Fraction f(int n)} - создаст дробь $\frac{n}{1}$
    \item \texttt{atmla::Fraction f(double n)} - создаст дробь, которая с точностьую до 1e-5 будет равна n
    \item Так же поддерживается потоковый ввод, например \texttt{std::cin {>}> f} - принимает строку, строка может быть двух видов:
    \begin{description}
        \item[Число] Целое число n - получится дробь $\frac{n}{1}$
        \item[Дробь] Строка виде {n/m} - получится дробь $\frac{n}{m}$(n, m - целые числа)
    \end{description}
\end{enumerate}

\section{Как взаимодействовать с объектами}
На данный момент существует всего 2 класса, которые лежат в \texttt{namespace atmla} (увековечил своё имя). Для создания матрицы используйте:
\\ \\
atmla::Matrix<{\textit{datatype}}> name(parameters);
\\ \\
\textit{datatype} может быть \texttt{int}, \texttt{double} или \texttt{atmla::Fraction}(я рекомендую использовать atmla::Fraction). Конструкторы бывают следующие:

\begin{enumerate}
    \item \texttt{matrix(int n, int m)} — создаёт пустую матрицу размером $n \times m$.
    \item \texttt{matrix(std::vector<std::vector<{\textit{datatype}}{>}> values)} — создаёт матрицу из вектора \texttt{values} (не знаю, зачем это нужно). Тип данных \texttt{values} может отличаться от типа данных матрицы (всё нормально приведётся).
    \item \texttt{matrix(int n, std::string special, \textit{datatype} lambda)} — создаёт специальную матрицу размером $n \times n$. \texttt{special} может принимать следующие значения:
    \begin{itemize}
        \item \texttt{"identity"} — создаёт единичную матрицу.
        \item \texttt{"zero"} — создаёт нулевую матрицу.
        \item \texttt{"jordan cell"} — Жорданова клетка с элементами \texttt{lambda} на диагонали (по умолчанию 0).
        \item \texttt{"diagonal"} — диагональная матрица с элементами \texttt{lambda} на диагонали (по умолчанию 0).
    \end{itemize}
\end{enumerate}

После создания матрицы её размеры изменить нельзя, но можно менять значения. Для этого предусмотрены следующие методы класса:

\begin{enumerate}
    \item \texttt{matrix.Set\_values\_from\_vector(std::vector<std::vector<{\textit{datatype}}{>}> values)} — тип данных в векторе должен совпадать с типом данных матрицы. Эта функция просто запишет данные из вектора в матрицу.
    \item \texttt{matrix.Set\_values\_manually()} — вводите все данные вручную.
    \item Так же поддерживается потоковый ввод маттрицы \texttt{std::cin {>}> matrix}
    \begin{description}
        \item[ВАЖНО] на данней момент есть небольшая проблема, \texttt{atmla::Fraction} можно вводить только по одному элементу в строке.
    \end{description} 
\end{enumerate}

\section{Функции вывода и операторы}
Матрицу можно вывести несколькими способами:
\begin{enumerate}
    \item Метод \texttt{matrix.print()} — просто выведет матрицу (почти бесполезно).
    \item Вывод в стандартный поток вывода:
    \begin{verbatim}
    std::cout << matrix;
    \end{verbatim}
\end{enumerate}

Существуют специальные способы вывода:
\begin{enumerate}
    \item Метод \texttt{matrix.print\_system(std::vector{\textit{datatype}} values)} — выведет матрицу и столбец \texttt{values} как СЛУ (система линейных уравнений). Тип данных в \texttt{values} может отличаться от типа данных матрицы.
    \item Метод \texttt{matrix.print\_matrix\_system(atmla::Matrix{\textit{datatype}} other)} — выведет систему матричных уравнений.
\end{enumerate}

Также перегружены стандартные операторы для матриц:
\begin{itemize}
    \item Сложение, вычитание, умножение матриц.
    \item Умножение, деление, сложение, вычитание на скаляр (каждый элемент матрицы будет поделен на скаляр).
\end{itemize}

\section{Другие функции}
\begin{enumerate}
    \item \texttt{matrix.Transpose()} — транспонирует матрицу.
    \item \texttt{matrix.Determinant()} — возвращает определитель матрицы.
    \item \texttt{matrix.Trace()} — возвращает след матрицы.
    \item \texttt{matrix.gauss(std::vector{\textit{datatype}} values)} — приводит матрицу к улучшенному ступенчатому виду, изменяя также \texttt{values}. После этого вызовите \texttt{matrix.print\_system(values)}, чтобы получить решение. Тип данных \texttt{values} может отличаться от типа данных матрицы.
    \item \texttt{matrix.Inverse()} — находит обратную матрицу.
\end{enumerate}

Наводя курсор на функцию, вы увидите информацию о её параметрах и возможных ошибках.

\section{Заключение}
После написания кода, в терминале зайдите в папку \texttt{builds}, выполните команду \texttt{cmake ..}, затем \texttt{cmake --build .}. После этого в папке \texttt{build} появится исполняемый файл \texttt{matrix.exe}, связанный с \texttt{main.cpp}.

Удачи!

\end{document}
